\documentclass[a4paper,12pt,titlepage]{report}
\usepackage[utf8]{inputenc}
\usepackage[T1]{fontenc}
\usepackage{lmodern}
\usepackage[a4paper]{geometry}
\usepackage[french]{babel}
\usepackage{amsmath}
\usepackage{mathcmd}
\usepackage{amssymb}
\usepackage{mathrsfs}
\usepackage{graphicx}
\usepackage{appendix}
\usepackage{hyperref}
\usepackage{subcaption}
\usepackage{setspace}
\usepackage[intoc]{nomencl}
%\usepackage{algorithm}
%\usepackage{listing}
\usepackage{verbatim}
\makenomenclature
\makeindex

\begin{document}


%==================================== Page de Garde ================================================ =======================================================================================================
\begin{titlepage}
 
	\begin{center}
	\begin{figure}[!h]
	\centering	
		\begin{subfigure}[b]{0.3\textwidth}
%		\includegraphics[height = 2cm, keepaspectratio]{graphes/mines_nancy.png}
		\end{subfigure}
		\begin{subfigure}[b]{0.3\textwidth}
		\includegraphics[height = 2cm, keepaspectratio]{graphes/elie_cartan.png}
		\end{subfigure}
		\begin{subfigure}[b]{0.3\textwidth}
		\includegraphics[height = 2cm, keepaspectratio]{graphes/univ_lorraine.png}
	\end{subfigure}
	\end{figure}
 
	\textsc{École nationale supérieure des Mines de Nancy}\\[2cm]
	\textsc{Rapport de projet 2A}\\[1cm]
	\textsc{Tancrède Cohet et Pierre Gauthier}\\[1cm]
 
	\begin{doublespace}
		{ \huge \bfseries{Mélange par un écoulement stationnaire de fluide à faible 				Reynolds}}\\[2cm]
	\end{doublespace}
	\textmd{Laboratoire : Institut Élie Cartan}\\[1cm]
	\textmd{Tuteurs : Pierre Brancher et Jean-François Scheid }\\[1cm]
 
	% Bottom of the page
	\vfill
	{\textit{{\large 21 Mars 2018}}}
 
	\end{center}
\end{titlepage}

%================= Table des matières =============================================================
\tableofcontents

\newpage



%======================== Introduction ===========================================================

\textbf{\Huge Introduction:}
\\
\\
\begin{onehalfspace}
La résolution d'équations à dérivées partielles sont la clé de voûte de la modélisation de systèmes physiques et financiers avec des conditions aux limites des systèmes. Cette problématique est particulièrement vraie en mécanique des fluides avec l'utilisation prépondérante de l'équation de Navier Stokes. Il est donc primordial d'avoir des moyens de résolution/simulation numérique de ce genre d'équation, qui respecte les différentes contraintes que posent la théorie des équations à dérivées partielles ( fonctions modélisées respectant les espaces de Sobolev). Une des méthodes numériques de résolution de Navier-Stokes est par exemple la méthode des éléments finis, sur maillage triangulaire. 
\newline
Le but de ce projet est d'étudier des écoulement chaotiques à faible nombre de Reynolds, c'est à dire des écoulements dont les forces visqueuses sont prépondérantes, qui sont présents typiquement dans l'industrie métallurgique ou dans l'industrie hydro-électrique. Cette problématique de résolution d'équations numériques à dérivées partielles est au cœur des enjeux du projet.
\newline
\newline
L'approche numérique eulérienne d'un mélange par advection chaotique traite du problème stationnaire avec champ de vitesse sinusoïdale qui correspond à un système de tourbillons orthogonaux, dont le modèle académique analytique a été étudié dans le cadre de la thèse de Valérie Toussaint, que nous allons reprendre avec deux tourbillons d'axes parallèles et un tourbillon qui leur est perpendiculaire.

Cette étude se fait dans la continuité d'un travail effectué par Ismail Mebsout et Oumaima Hammami durant l'année scolaire 2016-2017 où le but sera d'étudier les écoulements en s'affranchissant de certaines simplifications.
Dans la première partie de notre étude, nous allons présenter le problème, puis le mettre en équation et calculer la force magnétique composant le terme source de l'équation de Navier-Stokes, pour résoudre numériquement le problème et étudier les résultats.

%\end{onehalfspace}

%==========================================================================================================================================================
%												CHAPITRE 1 POSITION DU PROBLEME
%==========================================================================================================================================================
\chapter{Position du problème}

Pour générer l'écoulement que nous souhaitons étudier nous allons utiliser les forces de Laplace. Le fluide (de l'eau) considéré non visqueux dans une cuve est ionisé et placé sous influence d'aimants ferromagnétiques. Chaque particule du fluide va se mouvoir sous l'influence du champ magnétique. Notre but est de configurer correctement les champs magnétiques pour obtenir un écoulement chaotique. On peut définir l'advection chaotique comme un écoulement dans un fluide dont la trajectoire des points passe par tous les points du domaine, ou de manière mathématique avec les écoulements de Liapounov.
C'est à dire que théoriquement si on observe le passage d'une particule à travers une section du fluide au bout d'un temps infini, les points d'impact de la particule recouvreront de manière continue toute la section.
\subsection{Définition de l'advection chaotique}
Pour la définiton d'un écoulement chaotique nous allons considérer la différence entre les approches eulérienne et lagrangienne d'un écoulement. 
 Rappelons tout d'abord en se plaçant en eulérien , les variables indépendantes à considérer sont le temps t et la position $\vec{x}$ d'une particule fluide à l'instant t. \\
 L'étude de l'advection revient alors à étudier l'évolution au cours du temps du champ vectoriel passif diffusif $\vec{v}(\vec{x},t)$ régie par l'équation de convection-diffusion: 
 \[
 \frac{\partial \vec{x}}{\partial t} + \vec{u} \times \vec{\bigtriangledown}\vec{v}= D \vec{\Delta} v
\]
où D est la diffusivité moléculaire et $\vec{u}$ est le champ de vitesse advectant, incompressible et connu. \\
\\
Dans le cas où l'écoulement est laminaire, c'est à dire quand l'advection est prépondérante à la diffusion, l'écoulement étant complètement déterministe, le champ de vitesse $\vec{u}$ est une fonction régulière des coordonnées spatiales, et éventuellement du temps. 
\\ 
\\
En revanche, en se plaçant d'un point de vue lagrangien, les variables indépendantes sont alors le temps t et la position initiale $\vec{x_0}$ de la particule à l'instant $t_0$. Un tel système est un système dynamique à 3 degrés de liberté (nécessaire à un écoulement chaotique, il n'y a pas d'advection chaotique en 2D pour des écoulements stationnaires ce qui sera notre cadre d'étude), non autonome(si le champ de vitesse $\vec{v}$ dépend explicitement du temps, et généralement non linéaire à travers la dépendance spatiale du champ de vitesse, et pour lequel on différencie types de stabilité la où on ne le fait pas pour un systeme autonome). \\
\\
Dans le cas où le champ de vitesse est bi-dimensionnel et stationnaire , toutes les trajectoires de particules  fluides dérivent d'une fonction de courant régulière, on dit alors que le système est intégrable dans la terminologie des systèmes dynamiques. 
Dans les autres cas (notamment le notre) un tel système a de fortes chances d'être non intégrable, autrement dit de conduire à des trajectoires chaotiques. 
\\
\\
Ainsi on peut se trouver dans la situation d'un écoulement régulier dans la représentation eulérienne conduisant à une réponse essentiellement erratique si l'on considère l'advection d'un traceur d'un point de vue lagrangien : cette situation est appelée advection chaotique, ou chaos lagrangien.\\
L' écoulement chaotique va ainsi notament s'illustrer par une instabilité, notamment dans le cas de deux particules séparées par un écart $\delta x$, cet écart va croitre exponentiellement selon la théorie des exposants de Lyapunov. 
\\
Le chaos lagrangien est particulièrement intéressant lors de mélange laminaire, car il est bien plus intéressant que l'écoulement turbulent, il permet d'effectuer des mélanges pour un moindre coût d'énergie. En théorie à partir d'un temps infini, chaque particule aura parcouru l'intégralité des points de l'espace de référence.  On mettra en exergue cette assertion avec les sections de Poincaré, dont l'on explicitera les propriétés plus tard dans le rapport. 

\subsection{Dispositif expérimental} 

Expérimentalement, il est possible de réaliser cet écoulement à l’aide d’une distribution adéquate des forces. Pour cela, on utilise un système composé d'un parallélépipède, d’aimants permanents et d’électrodes pour générer un vecteur densité de courant. Le parallélépipède est un mélangeur dans lequel se trouve un liquide faiblement conducteur.
Afin de générer un tourbillon, on place deux aimants l’un en face de l’autre sur une moitié du parallélépipède pour créer un champ $\vec{B}$ vertical sur cette moitié. On place deux électrodes sur les deux autres faces opposées du parallélépipède pour créer un courant uniforme de densité $\vec{j}$.
\begin{figure}[!h]
	\begin{center}
	\centering	
%	\begin{subfigure}[b]{0.3\textwidth}
		\includegraphics[height = 7cm, keepaspectratio]{graphes/blender_cuve_champvec.png}
%		\includegraphics[height = 2cm, keepaspectratio]{graphes/axes.png}
		\caption{liquide faiblement ionisé soumis à un champ électrique $\vec{E}$, Blender}
		%		\end{subfigure}
	\end{center}
\end{figure}
\newline
On place un aimant de telle manière à générer un champ magnétique plus important sur une moitié du cube que l'autre.
\begin{figure}[!h]
	\begin{center}
	\centering
		\includegraphics[height = 8cm, keepaspectratio]{graphes/blender_cuve_mag2.png} 
		%\includegraphics[height = 2cm, keepaspectratio]{graphes/axes.png}
		\caption{lignes de champ magnétique, Blender}
	\end{center}
\end{figure}
\newpage
La force magnétique $\vec{f_l}=\vec{j_0}\land\vec{B}$ permet ainsi de générer un tourbillon.\newline
\begin{figure}[h]
	\begin{center}
	\centering	
%	\begin{subfigure}[b]{0.3\textwidth}
%		\end{subfigure}
%		\begin{subfigure}[b]{0.3\textwidth}
		\includegraphics[height = 10cm, keepaspectratio]{graphes/champvec2.png}
%		\includegraphics[height = 2cm, keepaspectratio]{graphes/axes.png}
		\caption{liquide faiblement ionisé soumis à un champ électrique et un champ magnétique $\vec{B}$, Blender}
%		\end{subfigure}
	\end{center}
\end{figure}
Afin de générer un tourbillon, on place deux aimants l'un en face de l’autre sur le deuxième tiers du parallélépipède pour créer un champ $\vec{B}$ vertical sur le tiers au milieu.
On place aussi deux électrodes sur les faces opposées du parallélépipède pour créer un courant uniforme de densité $\vec{j}$.
Une force de Laplace est générée, due a` la difference de potentiel (P1 - P0) en présence d’un champ magnétique $\vec{B}$, deux tourbillons sont alors générés dans le parallélépipède
%Pour obtenir la situation de 3 tourbillon dans une cuve avec 2 parallèles er le dernier perpendiculaires aux 2 autres, nous allons utiliser un liquide faiblement conducteur soumis à un champ électrique est à des champs magnétiques pour créer les tourbillon : On place trois aimants pour générer les trois tourbillons créant ainsi un mouvement d'advection chaotique.


\begin{figure}[!h]
	\begin{center}
	\centering
		\includegraphics[height = 10cm, keepaspectratio]{graphes/blender_cuve_champvec3.png} 
		%\includegraphics[height = 2cm, keepaspectratio]{graphes/axes.png}
		\caption{configuration avec 2 tourbillons, Blender}
	\end{center}
\end{figure}
\newpage

\subsection{Hypothèses de modélisation}
Nous devons d'abord modéliser notre système, car dans les cas de systèmes faisant intervenir des équations à dérivées partielles , la bonne définition du système est primordiale, avec notamment les conditions aux limites.\\
Notre système \{fluide\} sera modélisé par un fluide à faible nombre de Reynolds (visqueux) incompressible régi par les équations de Stokes que nous poserons dans le chapitre suivant. La densité de courant $\vec{j_0}$ est supposé uniforme sur le fluide.


%=====================================================================================================================================
%												CHAPITRE 1 CALCUL DU CHAMP MAGNETIQUE
%==========================================================================================================================================================
\newpage
\chapter{Modélisation du champ magnétique de l'aimant}
\subsection{Introduction}
Nous allons nous intéresser en premier lieu à modéliser le champ magnétique d'un aimant pour pouvoir obtenir à terme les forces de Laplace s'exerçant sur le fluide pour caractériser l'écoulement.

Nous cherchons ainsi à modéliser le champ d'induction magnétique $\vec{B}(x,y,z)$ induit par un aimant (Figure 1). L'objectif sera d'abord de déterminer le champ magnétique dans une configuration quelconque, pour ensuite exploiter ces résultats et les employer dans la résolution numérique de Navier Stokes. \\
\begin{figure}[h]
\begin{center}
\includegraphics[height =4 cm, keepaspectratio]{graphes/champ_aimant1.png} %on affiche figure aimant
\includegraphics[height = 2cm, keepaspectratio]{graphes/axes.png}
\caption{champ magnétique d'un aimant coupe 2D}
\label{figure 1}
\end{center}
\end{figure}

On suppose que l'aimant à une longueur suffisante selon $\vec{z}$ pour être considéré comme infini selon l'axe $\vec{z}$.
On en déduit donc par le principe de Curie que le champ magnétique est invariant par translation selon $\vec{z}$. 
Ainsi \[\vec{B}(x,y,z) = \vec{B}(x,y)\]

On modélise donc le champ magnétique en deux-dimensions dans cette partie, dans la suite nous considérerons que le champ d'induction aura la même valeur selon la troisième dimension.
%======================= Mise en forme du probleme ================================================== 
\section{Mise en forme du problème}

Nous allons rechercher une solution du champ magnétique dans le domaine de résolution $\Omega$ (figure 2), $\Omega \subset R^{2}$. \\
\begin{figure}[h]
\begin{center}
\includegraphics[height = 8cm, keepaspectratio]{graphes/Espacedetravail.png} 
\caption{domaines de résolution du problème, Blender}
\label{figure 2}
\end{center}
\end{figure}
On définit $\Omega_{int}$ le domaine de l'aimant, et $\Omega_{e}$ le domaine extérieur à l'aimant. 
\[\Omega_{e} = \Omega_{int} - \Omega\]

\newpage
On se place dans le régime stationnaire  étant donné qu'il n'y a pas de mouvements de l'aimant. \\
Le champ d'induction magnétique $\vec{B}$ est la somme du champs magnétique $\vec{H}$.

\[\vec{B}=\mu _{0}(\vec{H}+\vec{M})\]
$\mu _{0}$ est la perméabilité magnétique du vide. \\
L'aimantation $\vec{M}$ est nulle en dehors de l'aimant.
 
\[ 
\left\{
\begin{array}{ccc}
\begin{aligned}
	\vec{M} &= \vec{0} \  \text{dans} \ \Omega_{e} \\ 
	\vec{M} &= \vec{M_{0}} = \text{constante}  \ \text{dans}  \  \Omega_{\text{int}}
\end{aligned}
\end{array}
\right.
\]

Pour trouver le champ magnétique, on applique les équations fondamentales de la magnétostatique. L'équation de Maxwell nous donne 
\[
	\left\{
	\begin{array}{ccc}
	\begin{aligned}
		\Rot{\vec{H}} &= \vec {0} \\
		\Div{\vec{B}} &= 0
	\end{aligned}
	\end{array}
	\right.
\]

Le domaine étant simplement connexe, $\vec{H}$ dérive d'un potentiel $u$	.
\[
	\left\{
	\begin{array}{ccc}		
	\begin{aligned}
		\vec{H} &= \Grad{\vec{u}} \\
		\vec{B} &= \mu_{0}\times(\vec{\Grad{u}}+\vec{M})
	\end{aligned}
	\end{array}
	\right.
\]

Au sens des distributions pour toute fonction $\varphi\ $  dans $D(\Omega)$	
\[<\Div\vec{B},\varphi> \ = \ <-\vec{B},\vec{\Grad{\varphi}}>\]
On suppose $\vec{B} \in L^{3}_{1}(\Omega)$
\[<\Div\vec{B},\varphi> \ = \  -\int_{\Omega}\vec{B}. \vec{\Grad{\varphi}}\]
\[<\Div\vec{B},\varphi> \ = \ -\int_{\Omega}\mu_{0}(\vec{\Grad{u}}+\vec{M}) .{\Grad{\varphi}} = 0\]
Ainsi
\[\int_{\Omega}(\vec{\Grad{u}}+\vec{M}) . \vec{\Grad{\varphi}} = 0\]
Et comme $\vec{M}$ est nul en dehors du domaine $\Omega_{\text{int}}$ de l'aimant, nous obtenons (1)

On reconnait un problème de Dirichlet homogène qui admet une unique solution dans $H_{0}^{1}(\Omega)$ d'après le théorème de Lax-Migram.
\begin{equation}
\label{E}
\forall \varphi\ \in \Omega, \ \int_{\Omega}\bigtriangledown u .\bigtriangledown{\varphi} = -\vec{M_{0}}. \int_{\Omega_{int}}\bigtriangledown\varphi
\end{equation}

Adimensionnons (1)				:
\[
	\forall \varphi\  \in \Omega,\  \frac{1}{|\vec{M}_{0}|}\int_{\Omega}\bigtriangledown u.\bigtriangledown \varphi
	= -\vec{e}_{y}.\int_{\Omega_{int}}\bigtriangledown \varphi
\]
On prend pour la suite \[U =  \frac{u}{|\vec{M}_{0}|}  \] ce qui nous donne \[\vec{H^*} =\frac{\vec{H}}{M_0} \]
On établit ainsi \label{A}

\begin{equation}
\boxed{
\label{A}
	\forall \varphi\  \in \Omega,\ \int_{\Omega}\bigtriangledown U.\bigtriangledown \varphi
	=- \vec{e}_{y}.\int_{\Omega_{int}}\bigtriangledown \varphi}
\end{equation}
\\

\section{Méthode des éléments finis}

\subsection{Le maillage}
On recherche une solution approchée de l'équation numériquement en passant de l'espace continu à un espace discret. \\
\\
On utilise la méthode des éléments finis en recherchant une solution dans l'espace 
$V_{h} = \{u \ | \ u \in P^{1}(\Omega^{2}), u \in H^{1}_{0}(R^{2})\}$. \\
\\
On introduit une triangulation $T_{h}$ en subdivisant $\overline{\Omega}$, de bord $\Gamma \ = \ \partial\Omega$. 
Cette triangulation vérifie les propriétés suivantes :
\\
\begin{itemize}
  \item l’intersection de deux triangles K de $T_{h}$ doit être réduite à un sommet commun,\\ à une arête commune et entière ou à l'ensemble vide
  \item l’aire des triangles ne doit pas être nulle
  \item tous les coins du bord $\Gamma$ sont des sommets de triangles K de $T_{h}$
\end{itemize}

Le maillage ainsi construit est tel que
\[
\overline{\Omega} \ = \ \bigcup_{K \in T_{h}} K 
\]
\begin{figure}[h]
\begin{center}
\includegraphics[height = 8cm, keepaspectratio]{graphes/Maillage_initial.png} 
\caption{\label{figure 3 } maillage effectué sur matlab à l'aide de mesh2D d'un carré 1$\times$1.05 avec un pas constant h = 0.1}
\end{center}
\end{figure}

On note que le maillage est caractérisé par la longueur de la plus petite arrête $h_{min}$ telle que 
\[
h_{min} \ = \ \min_{K \in T_{h}} h_{K}
\]
$h_{min}$ va notamment déterminer l'erreur entre la solution continue et la solution discrète.
On choisit une base de $V_{h}$, on prend la famille des fonctions de base  \\ $\varphi_{1}, \varphi_{2}, ..., \varphi_{N_{h}}$  telles qu'elles valent 1 en un sommet d’un triangle, et 0 pour tous les autres sommets. On note $P_{1},P_{2}, ..., P_{N_{h}}$ les sommets des triangles du maillage.
Les fonctions de base sont donc définies ainsi :
\[
\forall i,j \in [1, N_{h}]^{2} \ \varphi_{i} \in L^{2}(\Omega),\ \varphi_{i}(P_{j}) \ = \ \delta_{ij} \ \
\varphi_{i} \ = \ 0 \ sur \ \Gamma \]
Le support de $ \varphi_{i}$ est la réunion de tous les triangles ayant pour sommet $P_{i}$.\\
On vérifie que la famille ($\varphi_{1}, \varphi_{2}, ..., \varphi_{N_{h}}$) est une base de $V_{h}$.
Ainsi toute fonction g dans $V_{h}$ peut s'écrire comme une combinaisons linéaire des $\varphi_{i}$
\[
g = \sum_{i=1}^{N_{h}}{\ g_{i}\varphi_{i}} \text{ \ \ où } g_{i} = g(P_{i}),\  i= 1,2,...,N_{h}
\]

%=====================================================================================================
\subsection{Mise en équation}
%=======================================================================================================

Réécrivons (1.2) dans $V_{h}$ : 

\[
	\forall \varphi \in V_{h} , \ \int_{\Omega}\bigtriangledown u_{h}. \bigtriangledown \varphi = -\vec{M_{0}}. \int_{\Omega_{int}}\bigtriangledown \varphi
\]
Nous pouvons écrire $u_{h} = \sum_{i=1}^{N_{h}}{\ U_{i}\varphi_{i}} \text{ \ \ où } U_{i} = U(P_{i}),\  i= 1,2,...,N_{h}$ \\ et choisir $\varphi\ =\ \varphi_{i}$
ainsi
\[
	\int_{\Omega}\bigtriangledown u_{h}.\bigtriangledown \varphi\ 
	=\ 
	\sum_{i, j \in [1, N_{h}]^{2}} \int_{\Omega}u_{i}\bigtriangledown\varphi_{i}. \bigtriangledown\varphi_{j} 
	= 
	\sum_{i,j \in [1, N_{h}]^{2}} A_{ij} u_{i}
\]
avec A la matrice $N_{h} \times N_{h}$ de coefficients
\[
A_{ij}  = \int_{\Omega}\bigtriangledown\varphi_{i}.\bigtriangledown\varphi_{j}
\]
et si on introduit le vecteur $\vec{f}$ de composantes $f_{1},\ f_{2},..., f_{N_{h}}$ définies par
\[
f_{j} =  -\vec{e_{y}}.\int_{\Omega} \bigtriangledown\varphi_{j}
\]
Alors l'équation (2) revient de manière équivalente à résoudre le système linéaire 
\[
	A \times U =F
\]
 avec 
\[
U = 
\begin{pmatrix}
   u_{1} \\
   u_{2} \\
   \vdots \\
   u_{N_{h}}
\end{pmatrix}
\ \ F = 
\begin{pmatrix}
   f_{1} \\
   f_{2} \\
   \vdots \\
   f_{N_{h}}
\end{pmatrix}
\quad
A_{ij}  = \int_{\Omega}\bigtriangledown \varphi_{i}. \bigtriangledown \varphi_{j}
\]
On appelle A la matrice de rigidité.  \\

%================= passage element de reference =====================================================
\newpage
\subsection{Passage à un élément de référence}
Pour les calculer numériquement les matrices, on utilise un triangle de référence normalisé $\hat{T}$. On passe du triangle $\Hat{T}$ à $T$ à l'aide de la transformation $F$. 
\begin{figure}[h!]
\begin{center}
\includegraphics[height = 6cm, keepaspectratio]{graphes/transformation_de_maillage.png} 
\caption{\label{figure 4 } passage à un maillage de référence}
\end{center}
\end{figure}
On a pour le calcul du gradient de $\varphi_{i}$
\[ \bigtriangledown_{\hat{x}} \hat{\varphi} = (J)^{T} \bigtriangledown_{x} \varphi  \]
où J est la matrice Jacobienne de la transformation.
Calculons la matrice Jacobienne qui va dépendre des coordonnées $(x_{i},y_{i})$ des sommets pour chaque triangle.
\[
x = \begin{pmatrix}
   x_{1} \\
   x_{2} 
\end{pmatrix}
= \begin{pmatrix}
   F_{1}(\hat{x}) \\
   F_{2}(\hat{x}) 
\end{pmatrix}
= \begin{pmatrix}
   \hat{x}_{1} \\
   \hat{x}_{2} 
\end{pmatrix}
= J \times \hat{x} + C
\]
On obtient ainsi à l'aide d'un changement de variable le calcul de la matrice de rigidité (voir annexe A) : 
\[
\boxed{
\begin{aligned}
A_{ij} = 
	\iint_{\Omega}\bigtriangledown_{x}{\varphi_{i}} \bigtriangledown_{x}{\varphi_{j}} &= 
	\sum_{T \in \text{Supp}(\varphi_{i})\times \text{Supp}(\varphi_{j})}	
	\frac{\text{aire}(T)^{2}}{4}
	\begin{pmatrix}
   		y_{3}-y_{1} &  	x_{1}-x_{3}\\
   		y_{1}-y_{2} &  x_{2}-x_{1}
	\end{pmatrix}
	^{2}
	\bigtriangledown_{\hat{x}} \hat{\varphi_{i}}
	\bigtriangledown_{\hat{x}} \hat{\varphi_{j}}
\end{aligned}}
\]
\\
\\
Calculons à présent le terme source $f_{j} =  -\vec{e_{y}}.\int_{\Omega} \bigtriangledown\varphi_{j}$.
\\
\\
D'après Green-Ostrogradsky
\[
	\iint_{\Omega}\bigtriangledown_{x}{\varphi_{i}} =\ 
	\int_{\partial\Omega_{\text{int}}}\varphi_{i}.\vec{\text{n}} \ \partial s
\]

Ainsi	
\[
	\begin{aligned}
		\int_{\partial\Omega_{\text{int}}}\varphi_{i}.\vec{\text{n}}\ \partial s &= 
		\sum_{e \in \partial\Omega_{\text{int}}}\ \int_{e}\varphi_{i}.\vec{\text{n}}_{e}\ \partial s \\ &=  
		\sum_{e \in \partial\text{supp}(\varphi_{i})}\ \int_{e}\varphi_{i}.\vec{\text{n}}_{e}\ \partial s
	\end{aligned}
\]
où $\vec{\text{n}}_{e}$ est le vecteur unitaire normal sortant à l'arête e. \\
\begin{figure}[h]
\begin{center}
\includegraphics[height = 4cm, keepaspectratio]{graphes/bord.png}
\caption{arêtes}
\label{figure 1}
\end{center}
\end{figure}
\[
	|e|\times \vec{\text{n}}_{e}
	=
	\begin{pmatrix}
		y_{j}-y_{i} \\
		x_{i}-x_{j}
	\end{pmatrix}
\]
avec le changement de variable $x = s\times \text{Pi} + (1-s)\times \text{Pj}$,\ \ $s \in [0;1]$
\[
	\int_{e}\varphi_{i}.\vec{\text{n}}_{e}\ \partial s = |e|(\int_{0}^{1}\varphi_{i}\ \partial x ).\vec{\text{n}}_{e}
\]
avec 
\[
	\int_{0}^{1}\varphi_{i}\ \partial s  = \text{aire du triangle rectangle de coté |e| et de hauteur 1} = \frac{|e|}{2}
\]
d'où
\[
	\int_{e}\varphi_{i}.\vec{\text{n}}_{e}\ \partial s =
	|e|(\int_{0}^{1}\varphi_{i}\ \partial x ).\vec{\text{n}}_{e} =
	\frac{|e|}{2}.\vec{\text{n}}_{e} = 
	\frac{|e|^{2}}{2}
	\begin{pmatrix}
		y_{2}-y_{1} \\
		x_{1}-x_{2}
	\end{pmatrix}	
\] 
Au final
\[
	\iint_{\Omega}\bigtriangledown_{x}{\varphi_{i}} =
	\sum_{e \in \partial \text{supp}(\varphi_{i})}\ \int_{e}\varphi_{i}.\vec{\text{n}}_{e}\ \partial s =
	\sum_{e \in \partial\text{supp}(\varphi_{i})}
	\frac{|e|}{2}
	\begin{pmatrix}
		y_{2}-y_{1} \\
		x_{1}-x_{2}
	\end{pmatrix}
\]
\[
	\iint_{\Omega}\bigtriangledown_{x}{\varphi_{i}} =
	\sum_{e \in \partial \text{supp}(\varphi_{i})}
	\frac{|e|}{2}
	\begin{pmatrix}
		x_{2}-x_{1} \\
		y_{1}-y_{2}
	\end{pmatrix}
\]
Au final 
\[\boxed{f_{j} =  -\vec{e_{y}}.\sum_{e \in \partial \text{supp}(\varphi_{i})}
	\frac{|e|}{2}
	\begin{pmatrix}
		x_{2}-x_{1} \\
		y_{1}-y_{2}
	\end{pmatrix}}
\]

				

%====================== Calcul du gradient =========================================================							
\subsection{Calcul du Gradient}

Les calculs précédents nous permettent de trouver après résolution le potentiel $u$, ce que nous voulons obtenir à l'issue de notre étude est le champ d'induction magnétique $\vec{B}$ que l'on trouve par $\vec{B}= \bigtriangledown u$.
\\
\newline  On écrit la forme variationnelle dans $V_{h}$ (ainsi $\vec{B}$ suit une décomposition dans $V_h$, tel que $\vec{B}=\sum_{i} B_{xi}\varphi_i:$

\[
\begin{aligned}
		\forall \varphi \in V_{h} , \ \int_{\Omega}B.  \varphi &= \int_{\Omega}\bigtriangledown u.\varphi  \\
		\sum_{i,j}\;  \sum_{T \in \text{Supp}(\varphi_{i})\times \text{Supp}(\varphi_{j})}B_{i}\int_{T} \varphi_i\varphi_j & =  \sum_{i}  \sum_{T \in \text{Supp}(\varphi_{i})\times \text{Supp}(\varphi_{j})} u_i \int_{T}\bigtriangledown\varphi_i\varphi_j  
\end{aligned}
\]
Nous pouvons écrire l'équation ci-dessus sous forme matricielle, en introduisant les matrices M et C, définit de la façon suivante :
\[
		M_{ij}=\int_{T} \varphi_i.\varphi_j \ \text{appellé la matrice de masse}
\]
\[
		C_{ij}=\int_{T} \bigtriangledown\varphi_i.\varphi_j 
\]
L'équation devient ainsi un problème linéaire sous forme matricielle  
\[
	M \times B=C \times U
\] 

Soit 
\[
	\boxed{B =M\up{-1}\times C \times U}
\]
Calculons ainsi les termes de la matrice C:
\[
\begin{aligned}
C_{ij} &= \iint_{\Omega}\bigtriangledown\varphi_i\varphi_j dx      
       = \sum_{K\in supp\varphi_j}\bigtriangledown_K\varphi_i \iint_{K} \varphi_jdx \\
	 &= \sum_{K\in supp\varphi_j}(J\up{-1}_K)^{T}(\bigtriangledown_{\hat{K}} \hat{\varphi_i})\iint_{\hat{K}}  |J_{K}| \hat{\varphi_j}\hat{dx} \\
	&= \sum_{K\in supp\varphi_j}(J\up{-1}_K)^{T}|J_K| (\bigtriangledown_{\hat{K} }\hat{\varphi_i})\iint_{\hat{K}}  \hat{\varphi_j}\hat{dx}
\end{aligned}
\]
\[\boxed{C_{ij} =  \sum_{K\in supp\varphi_j}(J\up{-1}_K)^{T}|J_K| (\bigtriangledown_{\hat{K} }\hat{\varphi_i})\iint_{\hat{K}}  \hat{\varphi_j}\hat{dx}} \]

Nous allons calculer pour l'implémentation algorithmique des matrices élémentaires qui vont regrouper sur chaque triangle les contributions croisées des trois nœuds, soit neuf termes dans chaque matrice élémentaire.
\newline
Calculons une matrice élémentaire de C sur un triangle K du maillage qui aura donc pour composantes (si on se place selon la composante x du gradient ) 

\[
|J_K| \times
\begin{pmatrix}
   	\bigtriangledown_{\hat{K} }{\varphi_1}_{|x}\iint_{\hat{K}}  \hat{\varphi_1}&\bigtriangledown_{K}{\varphi_1}_{|x}\iint_{\hat{K} } \hat{\varphi_2} &\bigtriangledown_{K}{\varphi_1}_{|x}\iint_{\hat{K} } \hat{\varphi_3}\\ 
  \bigtriangledown_{K }{\varphi_2}_{|x}\iint_{\hat{K }} \hat{\varphi_1} & \bigtriangledown_{K }{\varphi_2}_{|x}\iint_{\hat{K} }\hat{\varphi_2} & \bigtriangledown_{K }{\varphi_2}_{|x}\iint_	{\hat{K} } \hat{\varphi_3} \\
  \bigtriangledown_{K }{\varphi_3}_{|x}\iint_{\hat{K} } \hat{\varphi_1}&\bigtriangledown_{K }{\varphi_3}_{|x}\iint_{\hat{K} } \hat{\varphi_2} & \bigtriangledown_{K}{\varphi_3}_{|x}\iint_{\hat{K} } \hat{\varphi_3}
\end{pmatrix} 
\]
Calculons les intégrales :


\[	
  \left\{
    \begin{aligned}
      \iint_{\hat{K}} \hat{\varphi_1}{\hat{dx}} &= \int_{0}^{1}\int_{0}^{1-{\hat{x}}_2} (1-{{\hat{x}}_1}-{{\hat{x}}_2}) \hat{dx}_1 \hat{dx}_2 &=1/6\\
      \iint_{\hat{K} } \hat{\varphi_2}\hat{dx} &= \int_{0}^{1}\int_{0}^{1-{\hat{x}}_2} {{\hat{x}}_1} \hat{dx}_1 \hat{dx}_2 =1/6\\
       \iint_{\hat{K} } \hat{\varphi_3}\hat{dx} &= \int_{0}^{1}\int_{0}^{1-{\hat{x}}_2} {{\hat{x}}_1} \hat{dx}_1 \hat{dx}_2 =1/6\\
    \end{aligned}
  \right.
\]
\newpage
On a ainsi la matrice élémentaire du terme C sur un triangle K:

\[
\frac{1}{6}\times
	\begin{pmatrix}
   		\bigtriangledown_{K}{\varphi_1}_{|x} & \bigtriangledown_{K}{\varphi_1}_{|x} &  										    
   		\bigtriangledown_{K} {\varphi_1}_{|x} \\ 
    	\bigtriangledown_{K }{\varphi_2}_{|x} & \bigtriangledown_{K }{\varphi_2}_{|x} &  
    	\bigtriangledown_{K }{\varphi_2}_{|x} \\    										        
    	\bigtriangledown_{K}{\varphi_3}_{|x} & \bigtriangledown_{K}{\varphi_3}_{|x} & \bigtriangledown_{K}{\varphi_3}_{|x} \\ 
    \end{pmatrix} 
\]
\newline
Calculons également directement les matrices élémentaires pour la matrice de masse M définis par :
\[
M_{ij}=\iint_\Omega \varphi_i \varphi_j
\]
Sur un triangle K du maillage, on obtient de la même manière que précédemment 
\[
  \left.
    \begin{aligned}
		M_{ij}\up{K}=\int_0^{1}\int_0^{1-{\hat{x}}_2}|J_K|\hat{\varphi_i}(\hat{x})\hat{\varphi_j}(\hat{x})d\hat{x}\\
		=2\times\text{aire}(K)\times\int_0^{1-{\hat{x}}_2}\hat{\varphi_i}(\hat{x})\hat{\varphi_j}(\hat{x})d\hat{x}
    \end{aligned}
  \right.
\]
Il suffit donc de calculer les intégrales 
\[
	\int _0^{1}\int_0^{1-{\hat{x}}_2}\hat{\varphi_i}(\hat{x})\hat{\varphi_j}(\hat{x})d\hat{x}
\]
\\
par symétrie 
\[
	\forall i \int _0^{1}\int_0^{1-\hat{x}_{2}}\hat{\varphi_i}\up{2}= \int _0^{1} \int_0^{1-\hat{x}_{2}}\hat{x}_{2}^{2}d\hat{x}_{1}d\hat{x}_{2}=\frac{1}{12}
\]
\\%
\[
	\forall i \neq j \int _0^{1}\int_0^{1-{\hat{x}}_2}\hat{\varphi_i} \times \hat{\varphi_j}=\int _0^{1} \int_0^{1-\hat{x}_{2}}\hat{x}_{1}\hat{x}_{2}d\hat{x}_{1}d\hat{x}_{2}=\frac{1}{24}
\]
\\%
On en déduit la matrice de masse élémentaire sur un triangle de maillage 
\\%
\[
	\boxed{M\up{K}=\frac{\text{aire}(K)}{12}\times 
	\begin{pmatrix}
   		2 & 1 & 1 \\
   		1 &  2 & 1 \\
   		1 &  1 & 2
	\end{pmatrix}}
\]

%====================== Algorithme Matlab ======================================================= 
\subsection{Implémentation algorithmique}

Pour la résolution du problème, nous utiliserons matlab avec la bibliothèque Mesh2D par Darren Engwirda pour la création du maillage 2D.
La résolution est effectuée sur un maillage $5\times 5$ avec en son centre un aimant de taille $1\times 1$cm.

\begin{figure}[h]
	\begin{center}
	\includegraphics[height = 7cm, keepaspectratio]{graphes/maillage_resolution_champ_magnetique.jpg} 
	\caption{\label{figure 3 } maillage effectué sur matlab à l'aide de mesh2D dans la configuration décrite pour un pas h = 0.1}
	\end{center}
\end{figure}

Pour le calcul nécessaire des différentes matrices afin de résoudre le problème linéaire $A\times U = F$, les algorithmes reposent sur la même méthodologie de calcul :
Par exemple pour le calcul de la matrice A : 
\[
\begin{aligned}
A_{ij} = 
	\iint_{\Omega}\bigtriangledown_{x}{\varphi_{i}} \bigtriangledown_{x}{\varphi_{j}} &= 
	\sum_{T \in \text{Supp}(\varphi_{i})\times \text{Supp}(\varphi_{j})}	
	\frac{\text{aire}(T)^{2}}{4}
	\begin{pmatrix}
   		y_{3}-y_{1} &  	x_{1}-x_{3}\\
   		y_{1}-y_{2} &  x_{2}-x_{1}
	\end{pmatrix}
	^{2}
	\bigtriangledown_{\hat{x}} \hat{\varphi_{i}}
	\bigtriangledown_{\hat{x}} \hat{\varphi_{j}}
\end{aligned}
\]
Le principe consiste à ne pas recherche le support croisé des fonctions $\varphi_i$ et $\varphi_j$ (ce qui donnerait une complexité trop importante) mais à boucler sur les triangles et placer dans la matrice les contributions correspondantes à chacun de leurs nœuds avec leur indices. Notamment avec l'outil matriciel $sparse$ de matlab qui permet d'exploiter en temps de calcul et d'espace mémoire, le fait que la matrice de rigidité A est creuse, comme pour deux fonction $\varphi_{i}$ et $\varphi_{j}$, l'intersection de leurs supports se réduit à deux triangle.
\\
\\
Ainsi pour le calcul de A, chaque triangle engendre une matrice élémentaire $ 3\times 3$ (car 3 nœuds pour chaque triangle donc 9 termes croisés) symétrique étant donné le calcul du terme $A_{ij}$ dont on place les termes dans la matrice A à chaque itération sur les triangles.
\\
\\Le principe est le même pour le terme source, pour chaque triangle appartenant au bord de l'aimant avec comme noeuds sur le bord les nœuds $\text{p}_{i}$ et  $\text{p}_{j}$, on place deux fois le même terme $\frac{|e|}{2}\begin{pmatrix} 
   									0 & 1\\
								\end{pmatrix}^{T}.
								\vec{\text{n}}_{e}$ 
(que l'on à calculé p.21) aux indices $\text{p}_{i}$ et  $\text{p}_{j}$ dans la matrice du terme source.
\\
\\
On procède de la même manière pour le calcul des matrices M et C pour le gradient, dont nous avons explicité les matrices élémentaires.	

Voici une présentation succinte de l'algorithme de résolution, les fonctions utilisés sont détaillé dans l'annexe \ref{annexe_2}.

Pour la génération du maillage, on tulise la fonction meshfaces fournis par la bibliothèque Mesh2D par Darren Engwirda
\begin{verbatim}
 [v,t,fnum] = meshfaces(node,edge,face,hdata); %construction du maillage
 
 v : vecteur colonne regroupant les coordonnées des nœuds
 t : table de connectivité des triangle
 fnum : numéros de face des triangles du maillages (1 ou 2), 
 si sur le domaine des l'aimant ou de la cuve
\end{verbatim}
Calcul de la matrice de rigidité A et de la matrice de masse M : 
\begin{verbatim}
[M, nn, ibint, ic2] = matrixP1final(v,t,fnum,node1,edge1,node2,edge2);
[A]= matrixP1vect(v,t);
\end{verbatim}

On enlève les noeuds sur le bord du domaine de résolution comme la condition aux limites impose que $\Delta u = 0$ sur le bord, que nous considérons équivalente à $u = 0$ 
\begin{verbatim}
A1 = A1(ic2,ic2);	%ic2 sont les noeuds intérieurs du maillage
sol = A1\M;			%résolution linéaire
\end{verbatim}
Calcul du gradient de $u$ :
\begin{verbatim}
B = gradient(u,v,t);
 quiver(X,Y,B_X,B_Y);		%affichage du résultat final
\end{verbatim}

%====================== resultats  ===========================================================================
 
\subsection{Résultats de la simulation}

Voici les résultats de la simulation pour un maillage de pas 0.2 et de taille $5 \times 5$ pour un aimant de taille $1 \times 1$.
\begin{figure}[!h]
\begin{center}
\includegraphics[height = 10cm, keepaspectratio]{graphes/resultat_champ.jpg}
\caption{affichage résolution centrée sur [-2, 2]}
\label{figure 1}
\end{center}
\end{figure}
On voit que les vecteurs suivent les dessin des lignes de champ que l'on peut attendre d'un aimant.
On peut s'inquiéter de la discontinuité des valeurs au bord de l'aimant mais cela est en accord avec la formulation du champ d'induction comme la somme d'un champ d'excitation et d'une aimantation, nulle en dehors de l'aimant $\vec{B}=\mu _{0}(\vec{H}+\vec{M})$, d'où cette discontinuité que l'on observe.
\begin{figure}[!h]
\begin{center}
\includegraphics[height = 8 cm, keepaspectratio]{graphes/norme_du_champ.jpg}
\caption{affichage résolution centrée sur [-2, 2]}
\label{figure 1}
\end{center}
\end{figure}
%=========================================================================================================================================================
%												chapitre 2 equations de Navier_Stockes
%=========================================================================================================================================================
\newpage
\chapter{Équations de  Navier Stokes} 
Nous allons à présent nous intéresser à la modélisation de l'écoulement, qui suit les équations de Navier-Stokes d'après les hypothèses que nous avons émis précédemment. 
Nous supposons que nous avons à notre disposition deux aimants, qui sont perpendiculaires sur les côtés de la cuve, l'un produit un champ $B_1$, l'autre produit un champ $B_2$.
Le premier aimant va produire un tourbillon dans le fluide, le deuxième va produire deux tourbillons parallèles, qui sont orthogonaux au premier tourbillon . Tout cela dans le but d'obtenir un écoulement tourbillonnaire. 
\begin{figure}[!h]
\begin{center}
\includegraphics[height = 8 cm, keepaspectratio]{graphes/3_tourbillons.png}
\caption{les trois tourbillons que nous souhaitons obtenir avec notre dispositif expérimental}
\label{figure 1}
\end{center}
\end{figure}
%======================== Equations de Navier-stokes==========================================================
\newpage
\subsection{Équations de Stokes}

Le liquide est un conducteur placé dans un champ magnétique et parcouru par un courant supposé uniforme $\vec{j_0}$.
\normalsize 
Afin de pouvoir trouver l'équation de la trajectoire d’un point donné, on cherche tout d’abord à trouver la vitesse dans l'écoulement du fluide contenu dans un domaine $\Omega$.
\\A partir des équations de Navier-Stokes incompressibles, on détermine simultanément la vitesse  $V$ et la pression $P$. 
\\
\\
On émet les hypothèses suivantes :
\\
Dans un premier temps, nous pouvons négliger les forces d’inertie du fluide comme on a un faible nombre de Reynolds.
Le fluide visqueux est en mouvement le long d’une paroi solide fixe , ce qui nous donne une vitesse nulle sur le bord car 
$\vec{V_{\partial\Omega}}=\vec{V_{\text{paroi}}}=\vec{0}$
\\	
Finalement, on se place en régime stationnaire.
\begin{equation*}
  \left\{
    \begin{aligned}
      &\vec{0}=-\vec{\bigtriangledown}P +\mu\vec{\Delta}V +\vec{f}\;\text{dans }\Omega \\
      &\text{div}(\vec{V})=0\;\text{dans }\Omega \\     
      &\vec{V}=\vec{0}\;\text{sur }\partial\Omega
    \end{aligned}
  \right.
\end{equation*}

La force $\vec{f}$ correspond à la force de Laplace induite par le champ magnétique. La densité volumique de cette force est donnée par :
\[
\begin{aligned}
	\vec{f}=\vec{j_0}\land\vec{B}&=j_0\times M_0\times \vec{j}\wedge \vec{B^*} \\
	 \vec{B^*} =\frac{\vec{B}}{M_0}, 
	\ &\text{et} \ j_0=\frac{\Delta E}{L}\sigma
	\\
\text{avec} \quad
&\sigma : \text{la conductivité du fluide }
\\%
&\text{L: la longueur de base de la cuve} \\
&\Delta E : \text{la différence de potentiel}
\end{aligned} \\
\]
Ainsi on peut calculer la force magnétique pour un champ magnétique $\vec{B_1}$
\[
	\vec{B_1} = 
	\begin{pmatrix}
   		B_{1,x}\\
  		 B _{1,y}\\
  		 0
	\end{pmatrix} 
\]
\[
	\vec{f}=j_0\vec{e}_{x}
	\wedge\begin{pmatrix}
   		B_{1,x}\\
   		B _{1,y}\\
   		0
\end{pmatrix}=
j_0\times \begin{pmatrix}
  			0\\
   			0\\
    		B_{1,y}
			\end{pmatrix} 
\]

%==================================================== Adimensionnement de l'équation de Stokes ============================================================

\subsection{Adimensionnement de l'équation de Stokes}

\normalsize Afin de résoudre un seul système indépendamment des constantes, on adimensionne l'équation de Stokes. Le régime stationnaire sera pris en compte après adimensionnement.
\\%
Posons :
\begin{equation*}
\vec{V'}(x,t)=\frac{\vec{V}(x,t)}{V_0}\;;\vec{x'}=\frac{\vec{x}}{L_0}\; ; \vec{f'}(\vec{x'})=\frac{\vec{f}(\vec{x})}{j_0M_0}\;;P'=\frac{P}{P_0}\; ;t'=\frac{t}{\frac{L_0^2}{\mu}}\;
\end{equation*}
En injectant dans l'équation de Stokes d'évolution :
\begin{equation*}
  \left\{
    \begin{aligned}
    \rho\frac{\partial \vec{V}}{\partial t}=-\vec{\bigtriangledown} P + \mu \vec{\Delta}\vec{V}+\vec{f}
        \end{aligned}
  \right.
\end{equation*}
il en résulte :
\begin{equation*}
  \left\{
    \begin{aligned}
    \frac{\partial \vec{V'}}{\partial t}=-\frac{L_0 P_0}{\rho \mu V_0}\vec{\bigtriangledown} P '+  \vec{\Delta}\vec{V'}+\frac{L_0^2j_0M_0}{\mu V_0}\vec{f'}
        \end{aligned}
  \right.
\end{equation*}
On adimensionne par rapport à $V_0$ et $P_0$ de sorte que $\frac{L_0 P_0}{\rho \mu V_0}=1$ et $ \frac{L_0^2j _0M_0}{\mu V_0}=1$, avec le régime stationnaire ainsi établi:
\[
      \vec{0}=-\vec{\bigtriangledown}P' +\vec{\Delta}V' +\vec{f'}\;\text{dans }\Omega 
\]

%================= Formulation variationnelle =========================================================
\subsection{Formulation variationnelle}

Afin d'obtenir la formulation variationnelle de l'équation de Stokes, on passe dans l'espace des distributions de sorte que pour toutes fonctions test $\varphi$ on a 
\[
	\int_\Omega\vec{\bigtriangledown}P'\times \vec \varphi=\int_\Omega-\mu\vec{\Delta}V'\times \vec \varphi +\int_\Omega\vec{f'}\times \vec \varphi 
\]
En appliquant la formule de Green, on obtient :

\[
\forall \varphi \in D(\Omega) \quad  \int_\Omega\vec{\bigtriangledown}P'. \vec \varphi + \int_{\Omega}\bigtriangledown V'. \bigtriangledown \varphi = \int_\Omega\vec{f'}. \vec \varphi
\]
Précédemment nous nous étions placé dans l'espace $V_{h}$ d'approximation polynomiale.
Nous pouvons ainsi décomposer les champs de pression $P'$, la vitesse $V'$ et la force $\vec{f'}$ dans la base des $(\varphi_{i})$:

\begin{equation*}
  \left\{
    \begin{aligned}
 &V '= \sum_{i=1}^{N_{h}}{\ V_{i}\varphi_{i}} \text{ \ \ où } V_{i} = V(P_{i}),\  i= 1,2,...,N_{h}\\
 &P '= \sum_{i=1}^{N_{h}}{\ P_{i}\varphi_{i}} \text{ \ \ où } P_{i} = P(P_{i}),\  i= 1,2,...,N_{h}\\
 &f' = \sum_{i=1}^{N_{h}}{\ f_{i}\varphi_{i}} \text{ \ \ où } f_{i} = f(P_{i}),\  i= 1,2,...,N_{h}\\
\end{aligned}
  \right.
\end{equation*}$ $ \\ 
On prend $\varphi\ =\ \varphi_{i}$ un vecteur de la base
\begin{equation*}
  \left\{
    \begin{aligned}
	&\int_{\Omega}\bigtriangledown V_{h}.\bigtriangledown \varphi\ 
	=\ 
	\sum_{i, j \in [1, N_{h}]^{2}} \int_{\Omega}V_{i}\bigtriangledown\varphi_{i}. \bigtriangledown\varphi_{j} 
	= 
	\sum_{i,j \in [1, N_{h}]^{2}} A_{ij} V_{i}\\
	&\int_{\Omega}\bigtriangledown P_{h}.\varphi\ 
	=\ 
	\sum_{i, j \in [1, N_{h}]^{2}} \int_{\Omega}P_{i}\bigtriangledown\varphi_{i}. \varphi_{j} 
	= 
	\sum_{i,j \in [1, N_{h}]^{2}}C_{ij} u_{i}\\
	&\int_{\Omega} f. \varphi\ 
	=\ 
	\sum_{i, j \in [1, N_{h}]^{2}} \int_{\Omega}f_{i}\varphi_{i}. \varphi_{j} 
	= 
	\sum_{i,j \in [1, N_{h}]^{2}} M_{ij} f_{i}\end{aligned}
  \right.
\end{equation*}
On peut donc écrire la formulation variationnelle sous la forme d'un problème linéaire:
\begin{equation*}
  \left\{
    \begin{aligned}
	&-A\times V + C \times P = M	\times f \ \text{sur} \ \Omega\\
	& C\times V = 0 \ \text{sur} \  \partial\Omega\\
	\end{aligned}
  \right.
\end{equation*}

%====================== principe de superposition 
\subsection{Principe de superposition}

La linéarité de l'équation nous permet d'appliquer le principe de superposition est donc de décomposer la résolution du problème en deux résolutions : une pour chaque terme source et ensuite de les sommer.
Pratiquement nous allons résoudre le problème en deux temps : d'un coté pour l'aimant centré qui donne deux tourbillons et d'autre part pour l'aimant décentré qui donne un tourbillon pour ensuite sommer les champs de vitesse après transposition du champ pour l'aimant décentré afin de pouvoir se retrouver dans la configuration où les tourbillons sont perpendiculaires.
\begin{figure}[!h]
	\begin{center}
	\includegraphics[height = 8 cm, keepaspectratio]{graphes/config_centre.png}
	\caption{résolution pour un aimant décentré}
	\label{figure 1}
	\end{center}
\end{figure}
\begin{figure}[!h]
	\begin{center}
	\includegraphics[height = 8 cm, keepaspectratio]{graphes/champvec2.png}
	\caption{résolution pour un aimant centré après transposition du champ de vitesse pour que les tourbillons soient perpendiculaires}
	\label{figure 1}
	\end{center}
\end{figure}
%===================== implementation algorithmique =======================================
\newpage
\subsection{Implémentation algorithmique}

Nous allons à présent mettre en place les calculs précédents de manière numérique sur Matlab en maillant le domaine de résolution qui est ici la cuve qui accueille le fluide.  
\\ 
On pose $ L_x,  L_y,  L_z$ les dimensions de la cuve selon x, y, et z.
\newline
Le domaine $\Omega = [0, L_x]\times[0, L_y]\times[0, L_z]$ est ainsi discrétisé par un maillage uniforme défini par les points : 
\\

\[
\left\{
\begin{array}{ccc}
  x_{ij} = (\frac{i\times L_x}{N+1} ,\frac{j\times L_x}{N+1}),    i,j=0,1,......,N+1\\
  y_{ij} = (\frac{i\times L_y}{N+1} ,\frac{j\times L_y}{N+1}),    i,j=0,1,......,N+1  \\
 z_{ij} = (\frac{i\times L_z}{N+1} ,\frac{j\times L_z}{N+1}),    i,j=0,1,......,N+1  
\end{array}
\right.
\]
\begin{figure}[!h]
	\begin{center}
	\centering
		\includegraphics[height = 8cm, keepaspectratio]{graphes/maillage_droit.png} 
		%\includegraphics[height = 2cm, keepaspectratio]{graphes/axes.png}
		\caption{maillage du domaine de résolution de coté 1 en prenant 11 points dans les directions x, y, et z à l'aide de la fonction matlab $ndgrid$}
	\end{center}
\end{figure}

%============================= calcul champ de vitesse =====================================
\newpage
\subsection{Calcul des matrices du système linéaire}

L'objectif va être comme précédemment de garnir selon un bouclage sur la liste des cubes constitutifs de l'espace de travail, les matrices A et B,  de masse et de rigidité. En effet comme précédemment, nous allons nous appuyer sur des matrices élémentaires calculées sur les triangles afin de calculer les sommes $\sum_{i, j \in [1, N_{h}]^{2}} V_{i}\int_{\Omega}\bigtriangledown\varphi_{i}. \bigtriangledown\varphi_{j} $ et $\sum_{i, j \in [1, N_{h}]^{2}}P_{i} \int_{\Omega} \bigtriangledown\varphi_{i}. \varphi_{j} $. 
\\
On peut décomposer, avec les fonctions test $v_h=
\left(
\begin{array}{ccc}
  \varphi^(j)     \\
  0     \\
  0     
\end{array}
\right)$, puis $v_h=
\left(
\begin{array}{ccc}
  0                  \\
  \varphi^(j)      \\
  0     
\end{array}
\right)$, et enfin $v_h=
\left(
\begin{array}{ccc}
  0                 \\
  0                  \\
  \varphi^(j)     
\end{array}
\right)$,
 la matrice symétrique A  selon cette forme : 
 \[A= \nu \times
\left(
\begin{array}{ccc}
  \sum^3_{l=1} W^{l,l}+ W^{1,1} &  (W^{1,2})^T &   (W^{1,3})^T\\
 W^{1,2}  & \sum^3_{l=1}W^{l,l}+ W^{2,2}     &  (W{2,3})^T\\
  W^{1,3} & W^{2,3}  & \sum^3_{l=1}  W^{l,l}+ W^{3,3}
\end{array}
\right)
\]
 On prend comme notation $W_{i,j}^{k,l}=\int_\Omega \frac{\partial \varphi^{(i)}}{\partial x_k} \times \frac{\partial \varphi^{(j)}}{\partial x_l}dx$ et $B=
\left(
\begin{array}{ccc}
  B^(1)   \\
  B^(2)  \\
  B^(3)  
\end{array}
\right) $\\ avec $B_{ij}^{(l)}=\int_\Omega \frac{\partial \varphi^{(i)}}{\partial x_k} \times \frac{\partial \varphi^{(j)}}{\partial x_l}dx$.
\\
Le second membre vaut $F_h=
\left(
\begin{array}{ccc}
  F^{(1)} \\
  F^{(2)}   \\
  F^{(3)} 
\end{array}
\right) $avec $F_j{(l)} =\int_{\Omega} f_l \varphi^{(j)}dx$. 

L'algorithme va donc d'abord créer des matrices élémentaires comme précédemment , mais cette fois ci non plus sur le format de triangle en 2D ( i.e pour un triangle à trois sommet des matrices élémentaires de format (3,3)) mais pour des cubes avec 27 points de calcul , donc des matrices élémentaires de format 27*27
D'abord on doit calculer les matrices élémentaires de rigidité:
\\
\[W_{i,j}^{k,l}=\int_\Omega \frac{\partial \varphi^{(i)}}{\partial x_k} \times \frac{\partial \varphi^{(j)}}{\partial x_l}dx\]
où $l_i$ sont les fonctions de base de $V_h$ associées à l'élément K. Pour déterminer ces
%intégrales, on se ramène à l’élément de référence $\hat{K}$ en effectuant un changement de variable dans l’intégrale, avec la transformation affine
intégrales, on se ramène à l'élément de référence $\hat{K}$ en effectuant un changement de variable dans l'intégrale, avec la transformation affine
$F_K(\hat{x})=B_K \hat{x} + b_K$ qui transforme $\hat{K}$ en K.
\\
 En reprenant les travaux précédents sur les transformations entre les éléments de référence, on rappelle que
$\hat{\bigtriangledown}\hat{v}(\hat{x})= [DF_K(\hat{x})]^T\times \bigtriangledown v(x)$ avec $[DF_K(\hat{x})]$ la matrice jacobienne de $F_K$. On a $\bigtriangledown(x) = [B_K^{-1}]^T\hat{\bigtriangledown} \hat{v}(\hat{x})$. Le principe algorithmique de la matrice de rigidité 3D repose sur le même remplissage que la matrice de rigidité 2D, (que nous avons mis en annexe), c'est à dire de répartir les valeurs selon les intersections de supports à l'aide la fonction $sparse$ de Matlab (codé en C++ dans le cas 3D) .  
\\
On voit dans l'algorithme ci dessous cette construction en matrice élémentaire (avec DxDy par exemple ,qui correspond à la matrice élémentaire associée à tous les termes de forme   $\frac{\partial \varphi^{(i)}}{\partial x_k} \times \frac{\partial \varphi^{(j)}}{\partial x_l}$)
\begin{verbatim}
    point=t(k,:);
    indice=point';
    detB=((v(t(k,2),1)-v(t(k,1),1)).*(v(t(k,4),2)-v(t(k,1),2)).*(v(t(k,5),3)-v(t(k,1),3)));
    
    DxDxval=detB.*(1./(v(t(k,2),1)-v(t(k,1),1)).^2).*DxDx;
    ...
    DyDzval=detB.*(1./((v(t(k,4),2)-v(t(k,1),2)).*(v(t(k,5),3)-v(t(k,1),3)))).*DyDz;
    DxWval=detB.*(1./(v(t(k,2),1)-v(t(k,1),1))).*DxW;
    DyWval=detB.*(1./(v(t(k,4),2)-v(t(k,1),2))).*DyW;
    DzWval=detB.*(1./(v(t(k,5),3)-v(t(k,1),3))).*DzW;
    
    AddMatElem(A,indice,indice,2*nu*DxDxval+nu*DyDyval+nu*DzDzval);
    AddMatElem(A,indice+nv2,indice+nv2,nu*DxDxval+2*nu*DyDyval+nu*DzDzval);
    AddMatElem(A,indice+2*nv2,indice+2*nv2,nu*DxDxval+nu*DyDyval+2*nu*DzDzval);
\end{verbatim}
%================= implementation algo =============================================================
\subsection{Raccordement à la modélisation du champ magnétique}

Pour résoudre le système linéaire qui découle des équation de Navier-Stokes, nous devons reprendre notre travail précédent pour former le second terme $f$, matrice colonne regroupant les valeurs scalaires de la force magnétique aux différents points du maillage.

Nous devons donc faire le lien entre le maillage de résolution en trois dimensions pour la résolutions des équations régissant l'écoulement et le maillage en deux dimensions sur lequel on calcule le champ magnétique. 

Comme nous l'avons énoncé, nous supposons que le champ magnétique est invariant selon la troisième dimension, il est donc inutile de le recalculer pour chaque "couche" du maillage 3D, d'autant plus que le maillage 3D est uniforme. 
La méthodologie que nous allons donc appliquer consiste à calculer une unique fois le champ magnétique sur une seule couche de maillage (la couche z = 0 en pratique dans l'algorithme) et reprendre les valeurs du champ sur les autres couches.

Ce que nous faisons en pratique est de générer un grand maillage de résolutions 2D pour calculer le champ magnétique, ce qui permet d'atténuer les effets de bord subis par le champ magnétique. 
\newline
Nous utilisons ensuite la fonction Matlab $delaunay$ qui permet à partir du maillage rectangulaire, de définir un maillage triangulaire (comme nous avons travaillé auparavant sur un maillage de type triangulaire quelconque auparavant pour le calcul du champ magnétique).
Nous identifions ensuite sur ce maillage de résolution 2D un domaine correspondant à la cuve et un domaine correspondant à l'aimant, en fonction de la configuration choisie. Le maillage sur le domaine assigné à la cuve doit bien sur correspondre à une couche du maillage de résolution 3D utilisé pour la résolution des équations de Navier-Stokes.

\begin{figure}[!h]
	\includegraphics[height = 8cm, keepaspectratio]{graphes/maillage_resolution_total_centre.jpg}
	\caption{\label{figure 3 } configuration dans le cas de la résolution pour la configuration avec l'aimant centré}
\end{figure}
\begin{figure}[!h]
	\includegraphics[height = 8cm, keepaspectratio]{graphes/maillage_resolution_total_decentre.jpg}
	\caption{\label{figure 3 } configuration dans le cas de la résolution pour la configuration avec l'aimant décentré}
\end{figure}

\newpage
Algorithmiquement, pour assigner les différents domaines du maillage de résolution 2D généré, nous utilisons le vecteur label $fnum$ qui prend sur chaque triangle la valeur 1, 2, ou 3 en fonction du domaine qui lui a été assigné.
\newline
Maintenant  notre objectif est de pouvoir adapter les données extraites à l'algorithme de Stokes 3D. Le premier obstacle est de réindexer le numéro des triangles pour qu'il y ait une bonne consistance des données lorsque l'algorithme va les utiliser: il faut renuméroter les triangles dans l'ordre. 
\newline  	

Un autre point important est le degré d'interpolation des espaces polynomiaux sur lesquels nous travaillons. En effet nous avons résolu les équations du champ magnétique avec des éléments finis de degrés d'interpolation polynomiales 1 (éléments P1), alors que la résolution des équations de Navier-stokes nécessite des éléments finis P2. Il faut donc doubler le nombre de points d'interpolation du champ magnétique avant d'injecter le second membre $f$ dans l'algorithme de résolution des équations de Navier Stokes. Les données prennent alors une nouvelle structure suite à ce dédoublement des points, les matrices ne suivent plus le sens de parcours des données fixées avec la fonction de génération de maillage $ngrid$(qui est une généralisation sur la troisième dimension  de $meshgrid$). Ce doublement de point du maillage avec la génération des nouvelles tables de descriptions du maillage t et v sont faites par la fonction $mesh_3D$ fournies par M. Scheid que nous utilisons comme une boite noire.

Nous le voyons tout de suite avec un affichage avec un affiche $triplot$ après avoir appliqué $delaunay$ sur les données de la couche 2D du maillage 3D dont on a doublé les points.
\begin{verbatim}
z = 0;   % niveau z = 0
indz = find(v(:,3) == z);		% récupération des indices des nœuds pour le niveau 0
xz = v(indz,1); yz = v(indz,2); vz = [xz, yz];
tz = delaunay(xz,yz);               % triangulation de Delaunay 
triplot(tz,xz,yz);   				% affichage
\end{verbatim}

\begin{figure}[!h]
	\center
	\includegraphics[height = 8cm, keepaspectratio]{graphes/affichage_extraction_2D.jpg}
	\caption{\label{figure 3 } affichage avec $triplot$ du maillage extrait}
\end{figure}
Si on compare à une structure de données classiques générée avec $meshgrid$ ou $linspace$:
\begin{verbatim}
t_total_2D = delaunay(v_total_2D(:,1),v_total_2D(:,2)); 
triplot(t_total_2D ,X_total_2D,Y_total_2D);
xlim([4.5 5.5]);
ylim([4.5 5.5]);
\end{verbatim} 
\begin{figure}[!h]
	\center
	\includegraphics[height = 8cm, keepaspectratio]{graphes/affichage_extraction_2D_normal.jpg}
	\caption{\label{figure 3 } affichage avec $triplot$ du maillage extrait}
\end{figure}
\newpage
Pour comprendre la structure des données, il faut alors se rapporter  aux tables de connectivité pour le type de maillage 3D généré.
\begin{figure}[h]
\center
\includegraphics[height = 8cm, keepaspectratio]{graphes/table_de_connectivite.png} 
\caption{\label{figure 3 } table de connectivité, issue du cours $Analyse$ $numérique$ $des$ $équations$ $de$ $Navier$-$Stokes$, J.-F. Scheid}
\end{figure}
Nous comprenons alors que cette structure des données erratique est dûe à l'arrangement de la table de coordonnées v. Cela ne pose pas de problème dans la mesure où nous prenons garde à garder la même structure de données tout de long de notre algorithme de résolution est donc d'arranger selon la même structure le second membre $f$.
Et également pour pouvoir visualiser graphiquement nos résultat nous devons passer de cette structure particulière des données à une structure compatible à l'affichage avec la fonction matlab $quiver3$.
 
%================= résultats ====================================================================
\newpage
\subsection{Résultats}
Nous allons étudier graphiquement les différents résultats que nous avons obtenu et évaluer leur justesse en les interprétant physiquement par rapport à ce que nous attendrions avec les paramètres donnés.
\newline
Tout d'abord dans la configuration pour un aimant centré, nous observons sur la coupe clairement deux tourbillons symétriques de part et d'autre de l'aimant, ce qui est logique comme la configuration est symétrique selon le même plan de symétrie, et valide le bon placement des données dans la résolution du problème.
\\
Dans un premier temps, on discerne bien pour la première figure les deux tourbillons selon y, avec une forte vorticité au centre, car il s'agit du courant induit par les 2 tourbillons adjacents. 
\begin{figure}[!h]
\center 
\includegraphics[height = 8cm, keepaspectratio]{graphes/Paraview/coupe_aimant_centre.png} 
\caption{\label{figure 3 } coupe du champ de vitesse selon le plan XZ, pour Y petit (on est proche de l'aimant)}
\end{figure}
\newpage

\begin{figure}[!h]
\center 
\includegraphics[height = 8cm, keepaspectratio]{graphes/Paraview/flux_aimant_centre_100pt_length4.png} 
\caption{\label{figure 3 }flux dans cuve pour l'aimant}
\end{figure}
Sur cette figure-ci, nous discernons bien les lignes de champ du courant induit par les deux tourbillons, qui reboucle sur elle-même, mettant en exergue l'aspect stationnaire d'un point de vue eulerien. 
\begin{figure}[!h]
\center 
\includegraphics[height = 8cm, keepaspectratio]{graphes/Paraview/differents_flux_centre_100pt_lenght4_rad05.png} 
\caption{\label{figure 3 } différents flux selon les coupes prises dans la cuve, avec un aimant centré}
%\includegraphics[height = 8cm, keepaspectratio]{graphes/Paraview/coupe_2D_champ_centre.jpg} 
\caption{\label{figure 3 } champ magnétique selon XY avec un aimant centré}
\end{figure}

\newpage 
.
\newpage
Ces résultats sont en lien avec nos résultats de champ magnétique: contrairement aux hypothèses du projet précédent , le champ magnétique n'est effectivement pas constant dans la cuve, et s'évase vers les extrémités, on le remarque aussi d'ailleurs pour les autres lignes de flux, qui ont tendance à être courbées vis à vis des lignes de champ magnétique issues de l'aimant. 
Ensuite pour l'aimant décentré (avant la transposition du champ de vitesse) on observe
sur le plan de coupe près de l'aimant décentré un tourbillon centré selon z dans la cuve, respectivement 2 tourbillons selon y pour l'aimant centré. 
\begin{figure}[!h]
\center 
\includegraphics[height = 8cm, keepaspectratio]{graphes/Paraview/coupe_aimant_decentre.png} 
\caption{\label{figure 3 } coupe du champ de vitesse selon le plan XZ, pour Y petit (on est proche de l'aimant)}
\end{figure}

\begin{figure}[!h]
\center 
\includegraphics[height = 8cm, keepaspectratio]{graphes/Paraview/differents_flux_decentre_100pt_lenght4_rad05.png} 
\caption{\label{figure 3 }  différents flux selon les coupes prises dans la cuve, avec un aimant centré}
%\includegraphics[height = 8cm, keepaspectratio]{graphes/Paraview/coupe_2D_champ_decentre.jpg} 
\caption{\label{figure 3 } champ magnétique selon XY avec un aimant centré}
\end{figure}

\newpage
Enfin nous avons ici 2 coupes (une selon YZ, et une selon XZ) très proches des extrémités de la cuve.  On remarque que effectivement, si l'on se place du côté d'un aimant ou d'un autre, la prédominance de l'effet de l'aimant  se fait sentir. 
\newpage
\begin{figure}[!h]
\center 
\includegraphics[height = 8cm, keepaspectratio]{graphes/Paraview/total_coupe_pret_aimant_decentre.png} 
\caption{\label{figure 3 } coupe selon YZ pour le système complet}
\end{figure}
%\newline
\begin{figure}[!h]
\center 
\includegraphics[height = 8cm, keepaspectratio]{graphes/Paraview/total_coupe_pret_aimant_centre.png} 
\caption{\label{figure 3 } coupe selon XZ pour le système complet}
\end{figure}
\newpage

.
\subsection{Définition des sections de Poincaré}
Pour caractériser les propriétés chaotiques des trajectoires associées à ce champ de vitesse, nous allons utiliser l'outil classique de la théorie des systèmes dynamiques qu'est la section de Poincaré, dont l'un des atouts est qu'elle permet de diminuer la complexité des problèmes par une réduction du nombre de dimensions. 
\\
En mathématiques, particulièrement en système dynamique, une application de Poincaré est une application liée à une orbite périodique dans l’espace des états d’un système dynamique et un certain sous-espace de dimension moindre, appelé la section de Poincaré, transverse au  flot système.\\
 Plus précisément, on considère une orbite suffisamment proche d'une orbite périodique, avec une condition initiale sur la section de Poincaré, et on observe le point auquel cette orbite revient à la section pour la première fois, d'où son autre nom, application de récurrence.
\\
En fait, les intersections successives avec S notées Pn d’une trajectoire donnée peuvent être vues comme les itérées d’une condition initiale P0 par une application de S dans S:
\\
\[f: P_n \in S \to P_{n+1} = f(P_n) \in S \]
\\
Dans notre cas, les sections introduites sont des hyperplans, qui coupent l'espace selon les coupes de courant qui nous intéresse, pour déterminer si advection chaotique il y a.
\\
En effet , l'équation d'évolution de la particule fluide est:
\\
\[ \frac{d\vec{x}}{dt} = \vec{U}(\vec{x}), \]
où $\vec{x}(t)$ est la position de la particule fluide à l'instant t, et $\vec{U}$ est le champ de la vitesse, qui est stationnaire ( ne dépend pas du temps) , donc l'espace des phases correspond à l'espace physique. 
\\
On a deux cas:
\\
-Sur la surface $\Sigma $ apparaît une courbe régulière, qui correspond à un tore de fluide , ou quelques points , qui correspondent à des orbites théorique. Etant donné qu'il y a rebouclement des particules sur une même trajectoire, on qualifie la zone de semi-régulière, et n'est donc pas efficace pour le mélange. 
\\

-En revanche, si l'ensemble des des intersections entre les trajectoires et l'hyperplan $\Sigma$ est disséminé sur une partie de la section $\Sigma$, on en conclut que la trajectoire a un comportement erratique, caractérisant une région chaotique, qui est du coup une région de bon mélange. 
\newpage
\textbf{\Huge Conclusion:}
\normalsize
\\

L'objectif de ce projet a été, en partant de la thèse de Valérie Toussaint, qui a beaucoup développé autour de la théorie de l'advection chaotique, de d'abord modéliser finement l'espace physique d'étude des phénomènes d'advection chaotique, puis l'expérience permettant la génération de courants stationnaire lagrangiens à l'aide d'un fluide ayant un moment magnétique: nous devions d'abord simuler le champ magnétique sous l'hypothèse des éléments finis dans un espace de résolution de cuve. Puis avec les données du champ magnétique calculées sur un maillage particulier , une adaptation de ces dernières à la structure particulière en trois dimensions de l'algorithme de Stokes 3D a été  nécessaire. Enfin nous 
%\\
%\begin{onehalfspace}
%==========================================================================================================================================================
%														Annexe
%==========================================================================================================================================================

\begin{appendix}

%================================ matrice de rigidité ====================================================================
\chapter{Calcul de la matrice de rigidité}
\label{annexe_1}

La matrice Jacobienne de la transformation est définie par
\[	
J =
\begin{pmatrix}
  \frac{\partial\varphi_{1}}{\partial \hat{x}_{1}} & \frac{\partial\varphi_{2}}{\partial \hat{x}_{2}}  & \frac{\partial\varphi_{3}}{\partial \hat{x}_{1}}\\ 
  \frac{\partial\varphi_{1}}{\partial \hat{x}_{2}} & \frac{\partial \varphi_{2}}{\partial \hat{x}_{2}} & \frac{\partial\varphi_{3}}{\partial \hat{x}_{2}} 
\end{pmatrix} 
\times
\begin{pmatrix}
   x_{1} &  y_{1} \\
   x_{2} &  y_{2} \\
   x_{3} &  y_{3}
\end{pmatrix}
\]
\[
\frac{\partial \hat{\varphi}}{\partial \hat{x}_{1}}(\hat{x}) = 
\frac{\partial}{\partial \hat{x}_{1}}(\varphi(F(\hat{x})) = 
\frac{\partial \varphi}{\partial x_{1}}(x) \frac{\partial F_{1}}{\partial \hat{x}_{1}}(\hat{x}) +
\frac{\partial \varphi}{\partial x_{2}}(x) \frac{\partial F_{2}}{\partial \hat{x}_{1}}(\hat{x})
\]

Donc
\[ \bigtriangledown_{\hat{x}} \hat{\varphi} = 
\begin{pmatrix}
   \frac{\partial F_{1}}{\partial \hat{x}_{1}} & \frac{\partial F_{2}}{\partial \hat{x}_{2}}\\
\end{pmatrix}
\times 
\begin{pmatrix}
   \frac{\partial \varphi}{\partial x_{1}}(x) \\
   \frac{\partial\varphi}{\partial x_{2}}(x)
\end{pmatrix} = 
J^{T} \times \bigtriangledown_{x} \varphi \]

Calculons $\bigtriangledown_{\hat{x}} \hat{\varphi}$ sur le triangle de référence $\hat{T}$ :
\[
\left\{
\begin{array}{ccc} 
	\begin{aligned}
		\hat{\varphi}_{1} &= -\hat{x}_{1}+1-\hat{x}_{2} \\  %\qquad 	
		\hat{\varphi}_{2} &= \hat{x}_{1}                \\  %\qquad 
		\hat{\varphi}_{3} &= \hat{x}_{2}
	\end{aligned}
\end{array}
\right.
\]
En effet $\hat{\varphi}_{1}$ vaut 1 sur le sommet 1 et 0 sur les autres sommets (Figure 3).
Ainsi
\[ \bigtriangledown_{\hat{x}} \hat{\varphi}_{1} = 
\begin{pmatrix}
   -1 \\
   -1
\end{pmatrix}
\qquad
\bigtriangledown_{\hat{x}} \hat{\varphi}_{2} = 
\begin{pmatrix}
   -1 \\
   -1
\end{pmatrix}
\qquad
\bigtriangledown_{\hat{x}} \hat{\varphi}_{1} = 
\begin{pmatrix}
   -1 \\
   -1
\end{pmatrix}
\]
et donc
\[
\bigtriangledown_{\hat{x}} \hat{\varphi}(\hat{x}) = 
\begin{pmatrix}
   -1 & 1 & 0 \\
   -1 & 0 & 1 
\end{pmatrix}
\]
\[	
J =
\begin{pmatrix}
   -1 & 1 & 0 \\
   -1 & 0 & 1 
\end{pmatrix}
\times
\begin{pmatrix}
   x_{1} &  y_{1} \\
   x_{2} &  y_{2} \\
   x_{3} &  y_{3}
\end{pmatrix} =
\begin{pmatrix}
   x_{2}-x_{1} &  y_{2}-y_{1} \\
   x_{3}-x_{1} &  y_{3}-y_{1}
\end{pmatrix}
\]
Finalement,
\[
\begin{vmatrix}
   J
\end{vmatrix}
= (x_{2}-x_{1})\times (y_{3}-y_{1})-(x_{3}-x_{1})\times(y_{2}-y_{1})= 2\times\text{Aire}(T)
\]
où $|J|$ est le déterminant de la matrice $J$
et ainsi
\[ \bigtriangledown_{x} \varphi =  (J^{-1})^{T} \times \bigtriangledown_{\hat{x}} \hat{\varphi} \]
avec
\[
J^{-1} =  \frac{1}{|J|}
\begin{pmatrix}
   J_{22} & -J_{12} \\
   -J_{21} & J_{11}
\end{pmatrix}
\]
d'où
\[
*\begin{aligned}
	(J^{-1})^{T} 
	&=  	
	\frac{1}{|J|}
	\begin{pmatrix}
   		J_{22} & -J_{21} \\
   		-J_{12} & J_{11}	
	\end{pmatrix} \\
	&=
	\frac{1}{2\times \text{aire}(T)}
	\begin{pmatrix}
   		J_{22} & -J_{21} \\
   		-J_{12} & J_{11}	
	\end{pmatrix} 
	\end{aligned}
\]

Et le calcul du terme $A_{ij}$ de la matrice de rigidité, on obtient ainsi avec le changement de variable vers le triangle de référence.
\[
\begin{aligned}
	\text{A}_{ij} 
	&=
	\sum_{T \in \text{Supp}(\varphi_{i})\times \text{Supp}(\varphi_{j})} 
	\iint_{(i,j) \in T}\bigtriangledown_{x}{\varphi_{i}} \bigtriangledown_{x}{\varphi_{j}} \\
	&=
	\sum_{T \in \text{Supp}(\varphi_{i})\times \text{Supp}(\varphi_{j})}
	\iint_{(i,j) \in T} (J^{-1})^{T}|J|(J^{-1})^{T}|J|
	\times
	\bigtriangledown_{\hat{x}} \hat{\varphi_{i}}
	\times
	\bigtriangledown_{\hat{x}} \hat{\varphi_{j}} \\
	&=
	\sum_{T \in \text{Supp}(\varphi_{i})\times \text{Supp}(\varphi_{j})}	
	\frac{\text{aire}(T)^{2}}{4}
	\begin{pmatrix}
   		y_{3}-y_{1} &  	x_{1}-x_{3}\\
   		y_{1}-y_{2} &  x_{2}-x_{1}
	\end{pmatrix}
	^{2}
	\bigtriangledown_{\hat{x}} \hat{\varphi_{i}}
	\bigtriangledown_{\hat{x}} \hat{\varphi_{j}}
\end{aligned}
\]

%================================= CODE MATLAB ===========================================================================

\chapter{Codes Matlab}
\label{annexe_2}

\subsection{Modélisation du champ magnétique}









\iffalse
%==========================================================================================================================================================	
\chapter{Démonstration de l'existence d'une solution u}
%==========================================================================================================================================================
\label{Démonstration 1}
On a : 
\[
\left \{
\begin{array}{ccc}
  \Div\vec{B} = 0   &   \\
  \vec{B} = \mu_{0}\times(\vec{grad} +\vec{M}) \\
  \vec{M_{0} } \ sur \  \Omega_{i}  \ et \ \vec{0}  \ sur \  \Omega_{e}
\end{array}
\right .
 \iff \ \left\{\begin{array}{cc} -\Delta u= -div(\vec{M})=0 \ sur\ \Omega \\ \\u \ = \ 0 \ sur \ \partial \Omega \end{array}\right.
\]
\\
donc au sens des distributions , $\forall \varphi \in D(\Omega)$ 
\[
\begin{aligned}
<-\Delta u,\varphi> &=  0 \\
\iff \sum_{k=1}^{3} <  \frac{\partial u}{\partial x_{k}}, \frac{\partial \varphi}{\partial x_{k}}> &= 0
\end{aligned}
\]
\\
Or u $\in H^{1}_{0}(\Omega)$ donc $ \frac{\partial u}{\partial x_k} \in L^{2}(\Omega)$
\\
donc $<  \frac{\partial u}{\partial x_k}, \frac{\partial \varphi}{\partial x_k}>=\int_{\Omega} \frac{\partial u}{\partial x_k} \times \frac{\partial \varphi}{\partial x_k}\, \mathrm{d}x $
\\
donc le probleme peut s'établir ainsi: $<\bigtriangledown u ,\bigtriangledown \varphi>_{L^{2}(\Omega) }= 0 ,\ \forall \varphi \in D(\Omega) $
\\
Or on sait que l'ensemble $\Omega$ est bornée dans au moins une direction 
\\
donc on peut écrire $< u,\varphi>_{H^{0}_{1}(\Omega)}=0 \ \forall \varphi \in D(\Omega) $
\\
 0 est une forme lineaire continue pour n'importe quelle espace vectoriel 
 \\ donc d'apres le théorème de représentation de Riesz, dans $H^{0}_{1}(\Omega)$,
\\
$\exists ! u_{0} \in H^{0}_{1}(\Omega)$ telle que $<u_{0},v>_{H^{0}_{1}(\Omega)}=0,\ \forall v\in {H^{0}_{1}(\Omega)} $

Pour trouver le champ magnétique, on appliquer les équations fondamentales de la magnétostatique. L'équation de Maxwell nous donne 
\[
\begin{aligned}
\Rot{\vec{H}} &= \vec {0} \\
\Div{\vec{B}} &= 0
\end{aligned}
\]

Le domaine étant simplement connexe, $\vec{H}$ dérive d'un potentiel U.
\[\vec{H}= \Grad{\vec{U}}\]
\[\vec{B} = \mu_{0}\times(\vec{\Grad{U}}+\vec{M})\]

Au sens des distributions pour toute fonction $\varphi\ $ dans $D(\Omega)$
\[<\Div\vec{B},\varphi> =  <-\vec{B},\vec{\Grad{\varphi}}>\]
On suppose $\vec{B} \in L^{3}_{1}(\Omega)$
\[<\Div\vec{B},\varphi> =  -\int_{\Omega}\vec{B}. \vec{\Grad{\varphi}}\]
\[<\Div\vec{B},\varphi> =  -\int_{\Omega}\mu_{0}(\vec{\Grad{U}}+\vec{M}) .{\Grad{\varphi}} = 0\]
Ainsi
\[\int_{\Omega}(\vec{\Grad{U}}+\vec{M}) . \vec{\Grad{\varphi}} = 0\]
Et comme $\vec{M}$ est nul en dehors du domaine $\Omega_{\text{int}}$ de l'aimant, on obtient (1)
\fi



\iffalse
%==========================================================================================================================================================
\chapter{Démonstration équation (1)}
%==========================================================================================================================================================
\label{Démonstration 2}	
	Pour trouver le champ magnétique, on appliquer les équations fondamentales de la magnétostatique. L'équation de Maxwell nous donne 
	\[
		\left\{
		\begin{array}{ccc}
			\begin{aligned}
				\Rot{\vec{H}} &= \vec {0} \\
				\Div{\vec{B}} &= 0
			\end{aligned}
		\end{array}
		\right.
	\]

	Le domaine étant simplement connexe, $\vec{H}$ dérive d'un potentiel $u$	.
	\[
		\left\{
		\begin{array}{ccc}		
		\begin{aligned}
			\vec{H} &= \Grad{\vec{u}} \\
			\vec{B} &= \mu_{0}\times(\vec{\Grad{u}}+\vec{M})
		\end{aligned}
		\end{array}
		\right.
	\]

	Au sens des distributions pour toute fonction $\varphi\ $  dans $D(\Omega)$	
	\[<\Div\vec{B},\varphi> \ = \ <-\vec{B},\vec{\Grad{\varphi}}>\]
	On suppose $\vec{B} \in L^{3}_{1}(\Omega)$
	\[<\Div\vec{B},\varphi> \ = \  -\int_{\Omega}\vec{B}. \vec{\Grad{\varphi}}\]
	\[<\Div\vec{B},\varphi> \ = \ -\int_{\Omega}\mu_{0}(\vec{\Grad{u}}+\vec{M}) .{\Grad{\varphi}} = 0\]
	Ainsi
	\[\int_{\Omega}(\vec{\Grad{u}}+\vec{M}) . \vec{\Grad{\varphi}} = 0\]
	Et comme $\vec{M}$ est nul en dehors du domaine $\Omega_{\text{int}}$ de l'aimant, on obtient (1)
	%je rentre l'equation on pourra y faire appel plus tard avec \ref{E}
	\[
		\forall \varphi\ \in \Omega, \ \int_{\Omega}\vec{\Grad{u}}.\vec{\Grad{\varphi}} = -\vec{M_{0}}. \int_{\Omega_{int}}\vec{\Grad{\varphi}}
	\]
\fi

\end{appendix}
\end{onehalfspace}

\newpage
\Large \bf Bibliographie

\mdseries  "Introduction à l'analyse numérique"(1998), de Jacque Rappaz et Marco Picasso, publié par les Presses polytechniques et universitaires romandes.
\\
\\
"Méthode des éléments finis : élasticité plane" par Yves Dabard, Institut Universitaire de Technologie du Mans Département Génie Mécanique et Productique
, http://iut.univ-lemans.fr/ydlogi/index.html, 24 mars 2006 – 29 mars 2011
%
%" par un écoulement stationnaire de fluide à faible Reynolds", par Ismail Mebsout et Oumaima Hammami, 2017,  Institut ´ Elie Cartan de Lorraine.
%
\\
\\
\mdseries "Analyse numérique des équations de Navier-Stokes",de Jean-François Scheid,  Cours de Master 2 Mathématiques (Recherche) - Université de Lorraine, Nancy.
\\
\\
\mdseries "Projet de deuxième Année : Génération de maillages 2D avec Matlab" de Jean-Philippe LEBOUCHER Benjamin PACCOU avec comme chef de Projet Jonas KOKO, Institut
Supérieur d’ Informatique, de Modélisation et de leurs Applications
\bibliographystyle{plain}
\bibliography{bibli}
\end{document}








