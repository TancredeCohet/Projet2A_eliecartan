
\documentclass[a4paper,12pt,titlepage]{report}
\usepackage[utf8]{inputenc}
\usepackage[T1]{fontenc}
\usepackage{lmodern}
\usepackage[a4paper]{geometry}
\usepackage[french]{babel}
\usepackage{amsmath}
\usepackage{mathcmd}
\usepackage{amssymb}
\usepackage{mathrsfs}
\usepackage{graphicx}
\usepackage{appendix}
\usepackage{hyperref}
\usepackage{subcaption}
\usepackage{setspace}
\usepackage[intoc]{nomencl}
\makenomenclature
\makeindex

\begin{document}
%=======
%Copier coller à partir d'ici
\subsection{Définition de l'advection chaotique}
Pour la définiton d'un écoulement chaotique nous allons considérer la différence entre les approches eulériennes et lagrangienne d'un écoulement. 
 Rappelons  d'abord en se plaçant en eulérien , les variables indépendantes à considérer sont le temps t et la position $\vec{x}$ d'une particule fluide à l'instant t. \\
 L'étude de l'advection revient alors à étudier l'évolution au cours du temps du champ vectoriel passif diffusif $\vec{v}(\vec{x},t)$ régie par l'équation de convection-diffusion: 
 \[
 \frac{\partial \vec{x}}{\partial t} + \vec{u} \times \vec{\bigtriangledown}\vec{v}= D \vec{\Delta} v
\]
où D est la diffusivité moléculaire et $\vec{u}$ est le champ de vitesse advectant, incompressible et connu. \\
\\
Dans le cas où l'écoulement est laminaire, c'est à dire quand l'advection est prépondérante à la diffusion, l'écoulement étant complètement déterministe, le champ de vitesse $\vec{u}$ est une fonction régulière des coordonnées spatiales, et éventuellement du temps. 
\\ 
\\
En revanche, en se plaçant d'un point de vue lagrangien, les variables indépendantes sont alors le temps t et la position initiale $\vec{x_0}$ de la particule à l'instant $t_0$. Un tel système est un système dynamique à 3 degrés de liberté (nécessaire à un écoulement chaotique, il n'y a pas d'advection chaotique en 2D pour des écoulement stationnaire ce qui sera notre cadre d'étude), non autonome(si le champ de vitesse $\vec{v}$ dépend explicitement du temps, et généralement non linéaire à travers la dépendance spatiale du champ de vitesse, et pour lequel on différencie types de stabilité la où on ne le fait pas pour un systeme autonome). \\
\\
Dans le cas où le champ de vitesse est bi-dimensionnel et stationnaire , toutes les trajectoires de particules  fluides dérivent d'une fonction de courant régulière, on dit alors que le système est intégrable dans la terminologie des systèmes dynamiques. 
Dans les autres cas (notamment le notre) un tel système a de fortes chances d'être non intégrable, autrement dit de conduire à des trajectoires chaotiques. 
\\
\\
Ainsi on peut se trouver dans la situation d'un écoulement régulier dans la représentation eulérienne conduisant à une réponse essentiellement erratique si l'on considère l'advection d'un traceur d'un point de vue lagrangien : cette situation est appelée advection chaotique, ou chaos lagrangien.\\
L' écoulement chaotique va ainsi notament s'illustrer par une instabilité, notament dans le cas de deux particules séparé par un écart $\delta x$, cet écart va croitre exponentiellement selon la théorie des exposants de Lyapunov. 
\\
Le chaos lagrangien est particulièrement intéressant lors de mélange laminaire, car il est bien plus intéressant que l'écoulement turbulent, il permet d'effectuer des mélages pour un moindre cout d'énergie. En théorie à partir d'un temps infini, chaque particule aura parcouru l'intégralité des points de l'espace de référence.  On mettra en exergue cette assertion avec les sections de Poincaré, dont l'on explicitera les propriétés plus tard dans le rapport. 




\subsection{Définition des sections de Poincaré}
Pour caractériser les propriétés chaotiques des trajectoires associées à ce champ de vitesse, nous allons utiliser l'outil classique de la théorie des systèmes dynamiques qu'est la section de Poincaré, dont l'un des atouts est qu'elle permet de diminuer la complexité des problèmes par une réduction du nombre de dimensions. 
\\
En mathématiques, particulièrement en système dynamique, une application de Poincaré est une application liée à une orbite périodique dans l’espace des états d’un système dynamique et un certain sous-espace de dimension moindre, appelé la section de Poincaré, transverse au  flot système.\\
 Plus précisément, on considère une orbite suffisamment proche d'une orbite périodique, avec une condition initiale sur la section de Poincaré, et on observe le point auquel cette orbite revient à la section pour la première fois, d'où son autre nom, application de récurrence.
\\
En fait, les intersections successives avec S notées Pn d’une trajectoire donnée peuvent être vues comme les itérées d’une condition initiale P0 par une application de S dans S:
\\
\[f: P_n \in S \to P_{n+1} = f(P_n) \in S \]
\\
Dans notre cas, les sections introduites sont des hyperplans, qui coupent l'espace selon les coupes de courant qui nous intéresse, pour déterminer si advection chaotique il y a.
\\
En effet , l'équation d'évolution de la particule fluide est:
\\
\[ \frac{d\vec{x}}{dt} = \vec{U}(\vec{x}), \]
où $\vec{x}(t)$ est la position de la particule fluide à l'instant t, et $\vec{U}$ est le champ de la vitesse, qui est stationnaire ( ne dépend pas du temps) , donc l'espace des phases correspond à l'espace physique. 
\\
On a deux cas:
\\
-Sur la surface $\Sigma $ apparaît une courbe régulière, qui correspond à un tore de fluide , ou quelques points , qui correspondent à des orbites théorique. Etant donné qu'il y a rebouclement des particules sur une même trajectoire, on qualifie la zone de semi-régulière, et n'est donc pas efficace pour le mélange. 
\\

-En revanche, si l'ensemble des des intersections entre les trajectoires et l'hyperplan $\Sigma$ est disséminé sur une partie de la section $\Sigma$, on en conclut que la trajectoire a un comportement erratique, caractérisant une région chaotique, qui est du coup une région de bon mélange. 
\end{document}